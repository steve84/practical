\newcommand{\lecture}{Network Performance and Modelling\\
FS2013} 
\newcommand{\serieNumber}{Practical Exercise\\ Final Report} 
\newcommand{\authorsOfSerie}{Tobias Schmid 07-909-294 \\ Steve Aschwanden 05-480-686}


\documentclass[a4paper]{article} 
\usepackage{documentstyle}

\begin{document} 
%Header %
\pagestyle{fancy} %eigener Seitenstil
\fancyhf{} %alle Kopf- und Fußzeilenfelder bereinigen
\fancyhead[L]{\lecture} %Kopfzeile links
\fancyhead[C]{\serieNumber} %zentrierte Kopfzeile
\fancyhead[R]{\authorsOfSerie} %Kopfzeile rechts
\renewcommand{\headrulewidth}{0.4pt} %obere Trennlinie
\fancyfoot[C]{\thepage} %Seitennummer
\renewcommand{\footrulewidth}{0.4pt} %untere Trennlinie

 

\begin{center}
	\huge{Comparison of TCP and SCTP transport protocols over wireless multi-hop networks}
\end{center}

\section{Introduction}

\subsection{Stream Control Transmission Protocol}
Stream Control Transmission Protocol (SCTP) is a reliable, message oriented transport protocol that provides new services and features for IP communication. For the past twenty years, reliable communication service has been provided by TCP, unreliable by UDP. So, what has brought about the addition of a third protocol to the IP suite of protocols? Many of the features found in TCP and UDP can also be found in SCTP.

\subsubsection{Key Benefits}
SCTP improves upon TCP and UDP by integrating components of each. But the designers of SCTP did not stop there. There have been two new concepts added: multi-homing and multi-streaming.

\paragraph{Multi-homing}
SCTP was designed to handle the signalling of telecommunications over IP. Since  telecommunications are very susceptible to time delays, every millisecond counts. Multi-homing enables systems that have multiple interfaces, for redundancy, to use one over the other without having to wait. Within SCTP one interface is established as the primary and the rest become secondary. If the primary should fail for whatever reason, a secondary is selected and utilized. When the primary becomes available again, the communications can be transferred back without the application being aware there was an issue. While establishing the connections, the primary and secondary interfaces are checked and monitored using a heartbeat/heartbeat  acknowledgement process that validates addresses, and maintains a Round Trip Time (RTT) calculation for each address. The RTT can indicate that the primary is slower than a secondary and allow for the communications to migrate to the secondary interface.

\paragraph{Multi-streaming}
Using TCP, only one single data stream is allowed per connection. All of the information must be passed through that one stream. SCTP allows multiple simultaneous data streams within a connection or association. Each message sent to a data stream can have a different final destination, but each must maintain message boundaries. For example, systems cannot send parts of the same message through different streams; one message must go through one stream. When running an ordered data delivery system, if one of the packets is out of order or missing, the stream is blocked pending resolution to the order. This is called “Head-of-Line Blocking.” With the use of multi-streams, only the stream that is affected would be blocked; the other streams would continue to flow. By using multi-streaming with SCTP, the issue with web browsers only having the ability to handle two simultaneous connections goes away. The client or the web server could immediately open additional streams and send pictures, text, etc. through each stream, reducing overall latency. This could also reduce overhead that servers often incur with the numerous separate connections required to fulfil a request.

\paragraph{Selective acknowledgements}
In standard TCP, every message, or packet of information must be accounted for, resent as necessary, and processed in the order they were sent. SCTP has the ability to selectively acknowledge receipt of missing, disordered, or duplicated messages. Due to the nature of telecommunications most applications would end up discarding any unsynchronized messages. Therefore, the need to send and receive the information is forgone. This would mean that a portion of a word, a portion of a video, or a piece of the whiteboard refresh would be skipped over. The applications and users may notice a slight skip in the voice, video, or refresh. This is referred to as jitter within the telecommunications world and a small amount of jitter is often preferred to having the packet resent and reprocessed which would double the amount of jitter, usually making it more noticeable to the users.

\paragraph{Unordered data delivery}
Due to the very nature of networks not all packets may travel across the exact same path. If there is a time-delay using one path over another, the original messages could be out of order when received. Unordered data delivery allows for this instance and can correct the issue by reordering the messages correctly. Using TCP’s reliable data transfer feature requires that packets be processed in order. If one is missing or out of order, the packet must be reordered before processing can continue. SCTP allows for unordered data delivery and since it has multiple streams, only the one affected is temporarily blocked.

\subsection{Optimized Link State Protocol}
Optimized Link State Protocol (OLSR) is a proactive routing protocol, so the routes are always immediately available when needed. OLSR is an optimization version of a pure link state protocol. The topological changes cause the flooding of the topological information to all available hosts in the network. To reduce the possible overhead in the network protocol uses Multipoint Relays (MPR). The idea of MPR is to reduce flooding of broadcasts by reducing the same broadcast in some regions in the network.

\subsubsection{Control messages}
OLSR uses two kinds of the control messages: Hello and Topology Control (TC). Hello messages are used for finding the information about the link status and the host’s neighbours. With the Hello message the Multipoint Relay (MPR) Selector set is constructed which describes which neighbours has chosen this host to act as MPR and from this information the host can calculate its own set of the MPRs. The Hello messages are sent only one hop away but the TC messages are broadcasted throughout the entire network. TC messages are used for broadcasting information about own advertised neighbours which includes at least the MPR Selector list. The TC messages are broadcasted periodically and only the MPR hosts can forward the TC messages.
There is also Multiple Interface Declaration (MID) messages which are used for informing other host that the announcing host can have multiple OLSR interface addresses. The MID message is broadcasted throughout the entire network only by MPRs. There is also a “Host and Network Association” (HNA) message which provides the external routing information by giving the possibility for routing to the external addresses. The HNA message provides information about the network- and the netmask addresses, so that OLSR host can consider that the announcing host can act as a gateway to the announcing set of addresses. The HNA is considered as a generalized version of the TC message with only difference that the TC message can inform about route cancelling while HNA message information is removed only after expiration time.

\subsubsection{Multipoint relays}
The Multipoint Relays (MPR) is the key idea behind the OLSR protocol to reduce the information exchange overhead. Instead of pure flooding the OLSR uses MPR to reduce the number of the host which broadcasts the information throughout the network. The MPR is a host’s one hop neighbour which may forward its messages. The MPR set of host is kept small in order for the protocol to
be efficient. In OLSR only the MPRs can forward the data throughout the network. Each host must have the information about the symmetric one hop and two hop neighbours in order to calculate the
optimal MPR set. The two hop neighbours are found from the Hello message because each Hello message contains all the hosts’ neighbours. Selecting the minimum number of the one hop neighbours which covers all the two hop neighbours is the goal of the MPR selection algorithm.
Also each host has the Multipoint Relay Selector set, which indicates which hosts has selected the current host to act as a MPR. When the host gets a new broadcast message, which is need to be spread throughout the network and the message’s sender interface address is in the MPR Selector set, then the host must forward the message. Due to the possible changes in the ad hoc network, the MPR Selectors sets are updated continuously using Hello messages.

\subsubsection{Advantage}
OLSR is also a flat routing protocol, it does not need central administrative system to handle its routing process. The proactive characteristic of the protocol provides that the protocol has all the routing information to all participated hosts in the network. However, as a drawback OLSR
protocol needs that each host periodic sends the updated topology information throughout the entire network, this increase the protocols bandwidth usage. But the flooding is minimised by the MPRs, which are only allowed to forward the topological messages.


\end{document}